\documentclass[xcolor=dvipanames]{beamer}
\usetheme{Madrid}

\title{Anki-Pdf-Editor}
\subtitle{React + Spring}
\author{derMacon}

% to make a list of frames possible: 
% https://tex.stackexchange.com/questions/56512/is-there-any-way-to-produce-list-of-frames-with-beamer
\makeatletter
\newcommand\listofframes{\@starttoc{lbf}}
\makeatother
\addtobeamertemplate{frametitle}{}{%
  \addcontentsline{lbf}{section}{\protect\makebox[2em][l]{%
    \protect\usebeamercolor[fg]{structure}\textbullet\hfill}%
  \insertframetitle\par}%
}

\begin{document}
	\begin{frame}
		\titlepage
	\end{frame}
	
	\begin{frame}
		\listofframes
	\end{frame}
		
	\begin{frame}
		\frametitle{Introduction}
		This webapp displays a pdf-reader containing an editor to create digital index cards. These cards can then be reviewed with the open source program ankidroid. The whole purpose of this project was to design a more efficient workflow for creating those cards, since it always took quite some time to screenshot the required parts of the pdf document. To solve this issue I implemented a button / shortcut to directly copy the current page displayed in the pdf-viewer to the card.
	\end{frame}
	
	\begin{frame}
		\frametitle{Usage}
		\begin{itemize}
			\item To open run the programm run the command `java -jar \emph{NameOfJar}`. Once run, Ankidroid itself, the spring boot server and the react frontend should open automatically. 
		
			\item The Tomcat Server (Spring boot) will run on Port 8080 and the React frontend on Port 3000.
			
			\item To open a new Pdf document or select a new project open the dropdown menu on the \emph{new} button and select the corresponding item.
			
			\item To search for a word in the pdf // todo
			
			\item To paste the current page in one of the textareas (front-/backside) select the place where you want the image to appear and click the paste image button.
			
		\end{itemize}	
		
	\end{frame}
	
	\begin{frame}
		\frametitle{Shortcuts}
		// todo
	\end{frame}
	
	\begin{frame}
		\frametitle{Installation / Requirements}
		\begin{itemize}
			\item linux OS
			\item Ankiconnect Addon
		\end{itemize}
	\end{frame}
	
\end{document}
